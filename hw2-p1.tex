\begin{enumerate}
\item
  Let $\xi \in \bndrydom$ and $w(x)$ be a barrier on $\domain_1 \compactcont \domain$ with $w$ superharmonic in $\domain_1$, $w > 0$
  in $\closure{\domain}_1 \setminus \set{\xi}$, $w(\xi) = 0$.
  Show that $w$ can be extended to a barrier in $\domain$.

  (note $w$ is a local barrier in $\domain_1$, and $\zeta \in \domain_1$.
  So essentially this is straight from the notes for Perron's Method, though with more detail about why the harmonic function of any ball
  is at most the superharmonic function in the intersection of the ball and a small ball around $\xi$)

  Choose $r > 0$ s.t. $B = B_r(\xi) \compactcont \domain_1$, and define $m = \inf_{\domain_1 \setminus B}$.
  Then the function
  $$
  W(x) =
  \begin{cases}
    w_1(x) = \min(m, w(x)) & x \in B \cap \closure{\domain}\\
    w_2(x) = m & \text{otherwise}
  \end{cases}
  $$

  This is a barrier at $\xi$ in $\domain$.
  First we show that $W$ is continuous in $\domain$.
  Since the minimum of two continuous functions is necessarily continuous, $w_1$ is continuous in $B \cap \closure{\domain}$.
  $W$ is also continuous in $B^c \cap \closure{\domain}$, as it's just constant there.
  Finally, we note $\lim_{y \rightarrow z \in \bndry{B} \cap \closure{\domain}} W(y) = m$, since $\sup_{x \in B \cap \closure{\domain}} w_1(x) = m$.
  If not, then $\exists x \in \bndry{B}$ s.t. $w_1(x) < m$.
  But this contradicts the definition of $m$, so this is impossible, and thus, $W$ is continuous on $\domain$.

  Next we need to show that $W$ is superharmonic in $\domain$.
  First, we note that $w_1$ and $w_2$ are superharmonic in their domains; $w_1$ by definition, and $w_2$ because it's just a constant.
  Thus, to show that $W$ is superharmonic in $\domain$, we only need to consider every ball $B'$ which intersects the boundary of $B$,
  and show that any harmonic function $h$ in $B'$ with $h \leq w$ on $\bndry{B'}$ will have $h \leq w$ in $B'$ as well.

  For $x \in B' \cap B^c \cap \domain$, $W(x) = m = \sup_{y \in \domain} W(y)$.
  By the maximum priniple, since on $B'$ $h \leq m$, we have $h \leq m$.
  Thus, $h \leq W(x)$ for $x \in B' \cap B^c \cap \domain$.

  For $x \in B' \cap B \cap \domain$, since $h$ is harmonic, $h \leq W(x)$ on $\bndry{B' \cap B \cap \Omega}$,
  and $w_1$ is superharmonic in $B' \cap B \cap \Omega$, we have $h \leq W(x)$ in $B' \cap B \cap \Omega$.
  On $\bndry{B' \cap B}$, we have $h \leq W$, as $h \leq W$ on $\bndry{B'}$ and $h \leq m = W(y)$ for $y$ in $B^c \cap \Omega$.
  Then if we lift (or I guess lower?) $W$ to a harmonic function $g$ on $B$, $h \leq g \leq W$ on $B' \cap B \cap \Omega$.
  This follows from Lemma 2.2.3 from the notes on Perron's method.

  Thus, $W$ is superharmonic in $\domain$.
  Since $W(\xi) = w_1(\xi) = 0 \leq m$, $W(y) = m > 0$ on $\bndrydom \setminus B$, and $W(y) = w_1(y) > 0$ on $\bndrydom \cap B$,
  $W$ fulfills all of the requirements to be a barrier at $\xi$ relative to $\domain$.

\item
  Let $\domain = \set{x^2 + y^2 < 1} \setminus \set{-1 \leq x \leq 0, y = 0}$.
  Show that the function $w := -Re\left( \frac{1}{\ln(z)} \right) = - \frac{\log(r)}{(\log(r))^2 + \theta^2}$ is a local barrier at $\xi = 0$.

  First we need to show that there is a neighborhood $N$ s.t.
  for all $B \compactcont N$ and all harmonic $h$ in $B$ with $h(x) \leq w(x)$ for $x \in \bndry{B}$,
  we have $w \geq h$ for $x \in B$.

  This can be shown with complex analysis.
  First note that since $\domain$ is simply connected and excludes $0$, the logarithm exists on it, and is holomorphic.
  Next define $W(z) = \frac{1}{\ln(z)}$.
  $W$ is a composition of holomorphic functions on $\domain$, and thus, its real and imaginary parts are harmonic.
  Note that $w(x, y) = Re(W(x + i y))$ shows that $w$ is harmonic on $\domain$.
  From the notes on superharmonic functions, we know that harmonic functions are always superharmonic functions.
  Therefore, we can take any neighborhood $N \compactcont \domain$ we want and $w$ will be superharmonic on it.

  Next we need to show that in the closure of such a neighborhood, $w(x, y) > 0$ when $x^2 + y^2 \neq 0$.

  Since $0 < r^2 = x^2 + y^2 < 1$, $\log \left( r \right) < 0$.
  Then since $(\log(r))^2 + \theta^2 > 0$, $-\frac{\log(r)}{(log(r))^2 + \theta^2} > 0$.
  Since this is true for $\domain$, it's true for $N$ as well.

  Finally we note that $0 \leq \lim \limits_{r \rightarrow 0+} \left| \frac{\ln(r)}{(\ln(r))^2 + \theta^2} \right| \leq \lim \limits_{r \rightarrow 0+} \frac{1}{|\ln(r)|} = 0$,
  so we must have $w(0) = 0$ for this to be continuous.

  Thus, $w$ is a local barrier for $\xi = 0$ in $\domain$.
\end{enumerate}
